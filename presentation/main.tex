%%%%%%%%%%%%%%%%%%%%%%%%%%%%%%%%%%%%%%%%%%%%%%%%%%
%%%%%%%%%%%%%%%%%%%%%%%%%%%%%%%%%%%%%%%%%%%%%%%%%%
%%
%% Based one the "beamer-greek-two" template provided
%% by the Labo/
%% Adapted by Fabian Gröger, June 2020
%%
%%%%%%%%%%%%%%%%%%%%%%%%%%%%%%%%%%%%%%%%%%%%%%%%%%
%%%%%%%%%%%%%%%%%%%%%%%%%%%%%%%%%%%%%%%%%%%%%%%%%%
%%
\PassOptionsToPackage{unicode}{hyperref}
\PassOptionsToPackage{naturalnames}{hyperref}

% to use normal slides ratio
%\documentclass{beamer}

% to use 16:9 ratio
\documentclass[aspectratio=169, professionalfonts]{beamer}

%\usepackage{babel}
%\usepackage[utf8]{inputenc}


%%% FONT SELECTION %%%%%%%%%%%%%%%%%
%%% sans font %%%%%%%%%%
\usepackage{kmath,kerkis}
%\usepackage[default]{gfsneohellenic}

%%% Times NR %%%%%%%%%%
%\usepackage{newtxtext,newtxmath}
%%%%%%%%%%%%%%%%%%%%%%%%%%%%%%%%%%%%

\usepackage{color}
\usepackage{amsmath}
\usepackage{amssymb}

\usepackage{pgfgantt}
\usepackage{adjustbox}

%\usepackage{media9}
\usepackage{multimedia}

\usepackage{hyperref}
\hypersetup{
    colorlinks=true,
    linkcolor=black,
    filecolor=hslu_pink,
    urlcolor=hslu_pink,
}

%% Listings Paket ------------------------------------------------------
%%% Doc: ftp://tug.ctan.org/pub/tex-archive/macros/latex/contrib/listings/listings-1.3.pdf
\usepackage{listings, ../latex-lib/listings-rust/listings-rust}

\definecolor{codegreen}{rgb}{0,0.6,0}
\definecolor{codegray}{rgb}{0.5,0.5,0.5}
\definecolor{codepurple}{rgb}{0.5,0,0.33}
\definecolor{codepurblue}{rgb}{0.16,0.0,1.0}
\definecolor{backcolour}{rgb}{0.95,0.95,0.92}

\lstset{
    basicstyle =\ttfamily\color{black}\small, % Standardschrift
    commentstyle=\color{codegreen},
    keywordstyle=\bfseries\color{codepurple},
    numberstyle=\tiny\color{codegray},
    stringstyle=\color{codepurblue},
    numbers = left,              % Ort der Zeilennummern
    tabsize=2,              % Groesse von Tabs
    breakatwhitespace=false,              % An Leerzeichen umbrechen
%showspaces=true,			  % Leerzeichen anzeigen
    backgroundcolor=\color{backcolour},      % % Hintergrundfarbe der Listings
    breaklines=true,
    captionpos=b,
    keepspaces=true,
    numbersep=5pt,
    showspaces=false,
    showstringspaces=false,
    showtabs=false,
}

% Code auschnitt importieren aus datei
% example:
%\mylisting{from}{to}{language}{file}{descr}{path}
\newcommand{\mylisting}[6]{
    \lstinputlisting[language=#3,
        firstnumber=#1,
        firstline=#1,
        lastline=#2,
        caption={#4, #5},
        label={implementation_listing_#4_#5}]
    {#6}
}

%% End Listings Paket ------------------------------------------------------


% Have subfigures and captions
\usepackage{subcaption}
\usepackage{caption}

% Tikz to crate diagrams, thanks to: https://github.com/mvoelk/nn_graphics
% Start of tikz settings
\usepackage{tikz}
\usetikzlibrary{positioning,arrows.meta}
\usetikzlibrary{matrix, chains, positioning, decorations.pathreplacing, arrows}
\usetikzlibrary{shapes,arrows,positioning,calc,chains,scopes}

\usepackage{ifthen}
\usepackage{pgfplots}
\pgfplotsset{compat=1.16}
\pgfplotsset{every axis/.append style={tick label style={/pgf/number format/fixed},font=\scriptsize,ylabel near ticks,xlabel near ticks,grid=major}}

\usepackage{amsmath}
\DeclareMathOperator{\sigm}{sigm}
\newcommand{\diff}{\mathop{}\!\mathrm{d}}

% colors
\definecolor{snowymint}{HTML}{E3F8D1}
\definecolor{wepeep}{HTML}{FAD2D2}
\definecolor{portafino}{HTML}{F5EE9D}
\definecolor{plum}{HTML}{DCACEF}
\definecolor{sail}{HTML}{A3CEEE}
\definecolor{highland}{HTML}{6D885A}

\tikzstyle{signal}=[arrows={-latex},draw=black,line width=1pt,rounded corners=4pt]

\usepackage{epstopdf}
\usepackage{graphicx}
\graphicspath{{./images/}}

% to make beautiful tables
\usepackage{booktabs}

% appendix for beamer
\usepackage{appendixnumberbeamer}

% notes on beamer template
% when using notes, make sure to have a pdf viewer, which can use the notes
% for example: https://github.com/Cimbali/pympress/
\usepackage{pgfpages}
%\setbeameroption{show notes}
%\setbeameroption{show notes on second screen=right}

%% Debugging
%\usepackage{showframe}

%%
% load HSLU thesis layout
\usepackage{HSLU_Thesis_Beamer_Layout}
\setTeipelLayout{}% options: "draft" -> Watermark

\setcounter{tocdepth}{1}
%\beamertemplatenavigationsymbolsempty
\setbeamertemplate{headline}{}

%%%%%%%%%%%%%%%%%%%%%%%%%%%%%%%%%%%%%%%%%%%%%%%%%%%%%%%%%%%%
% Thesis Info %%%%%%%%%%%%%%%%%%%%%%%%%%%%%%%%%%%%%%%%%%%%%%
%%%%%%%%%%%%%%%%%%%%%%%%%%%%%%%%%%%%%%%%%%%%%%%%%%%%%%%%%%%%
% title
\title[PCP-Rust]{Programming Concepts \& Paradigms\\ Team-Projekt: Rust}
% author
\author[Schilter \& Preuß]{Roman Schilter \& Jan-Henrik Preuß}
% supervisor
\supervisor{Supervisor}{Marcel Baumann \& Ruedi Arnold}
% date
\presentationDate{May 31, 2024}
%%%%%%%%%%%%%%%%

\begin{document}

% typeset front slides
    \typesetFrontSlides

%%%%%%%%%%%%%%%%
% Start of Slides:

%%%%

    \begin{frame}{Project overview}
        \framesubtitle{This is a subtitle}
        \begin{itemize}
            \item What is rust
            \item Some detail on when it was created and other key facts here
        \end{itemize}

        \note{
            Here you can add notes to the slides
        }
    \end{frame}

%%%%

    % Example import from file
    \begin{frame}{Fibonacci in rust}
        Fibonacci example

        \mylisting{22}{31}{Rust}{Rust}{Fibonacci recursive}{../rust/exercise-prolog-w3-1/src/main.rs}

        \note{
            Still notes
        }
    \end{frame}

    % Example direct code, note the [fragile] annotation
    \begin{frame}[fragile]{Direct filter\_map\_reduce}
        Map and reduce example

        \begin{lstlisting}[language=Rust,escapechar=@,label={lst:map_reduce-test}]
fn filter_map_reduce(list: [&str; 4]) -> String {
    list.iter()
        .filter(|e| e.starts_with("T"))
        .map(|e| e.to_uppercase())
        .collect::<Vec<String>>().join(" ")
}\end{lstlisting}

        \note{
            Still notes
        }
    \end{frame}

%%%%

    \begin{frame}{Related work}
        \framesubtitle{Related work subject \#1}
        \begin{columns}[T]

            \column{0.5\textwidth}
            column left

            \column{0.5\textwidth}
            column right
        \end{columns}

        \note{
            Notes....
        }

    \end{frame}

    \begin{frame}{Questions and answers}
        \begin{center}
        {\fontsize{40}{50}\selectfont Thank You! \\[10pt] Q \& A}
        \end{center}
    \end{frame}

%% REFERENCES
% allowframebreaks: creates multiple slides if it is to long for one
    \begin{frame}[allowframebreaks]{References}
        \begin{thebibliography}{}
            \setbeamertemplate{bibliography item}[book]
            \bibitem{DeepLearning}
            Goodfellow, Ian, Yoshua Bengio, and Aaron Courville.
            \newblock \emph{Deep Learning}.
            \newblock MIT Press, 2016.

            \setbeamertemplate{bibliography item}[article]
            \bibitem{FaceNet}
            Schroff, Florian, Dmitry Kalenichenko, and James Philbin.
            \newblock \emph{FaceNet: A Unified Embedding for Face Recognition and Clustering}, 2015.
            \newblock \url{https://arxiv.org/abs/1503.03832}

            \setbeamertemplate{bibliography item}[online]
            \bibitem{DCASE}
            Dekkers, Lauwereins, Thoen et al.
            \newblock \emph{DCASE Challenge 2018 Task 5}, 2018.
            \newblock \url{http://dcase.community/challenge2018/task-monitoring-domestic-activities}
        \end{thebibliography}
    \end{frame}

%% APPENDIX

%%%% BACKUP
    \begin{frame}{Backup Slides}
        \framesubtitle{backup slides}
        Backup!
    \end{frame}

%%
\end{document}
