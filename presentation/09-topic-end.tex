% Final slides here

\begin{frame}{Related work}
    \framesubtitle{Related work subject \#1}
    \begin{columns}[T]

        \column{0.5\textwidth}
        column left

        \column{0.5\textwidth}
        column right
    \end{columns}

    \note{
        Notes....
    }

\end{frame}

\begin{frame}{Questions and answers}
    \begin{center}
    {\fontsize{40}{50}\selectfont Thank You! \\[10pt] Q \& A}
    \end{center}
\end{frame}

%% REFERENCES
% allowframebreaks: creates multiple slides if it is to long for one
\begin{frame}[allowframebreaks]{References}
    \begin{thebibliography}{}
        \setbeamertemplate{bibliography item}[book]
        \bibitem{DeepLearning}
        Goodfellow, Ian, Yoshua Bengio, and Aaron Courville.
        \newblock \emph{Deep Learning}.
        \newblock MIT Press, 2016.

        \setbeamertemplate{bibliography item}[article]
        \bibitem{FaceNet}
        Schroff, Florian, Dmitry Kalenichenko, and James Philbin.
        \newblock \emph{FaceNet: A Unified Embedding for Face Recognition and Clustering}, 2015.
        \newblock \url{https://arxiv.org/abs/1503.03832}

        \setbeamertemplate{bibliography item}[online]
        \bibitem{DCASE}
        Dekkers, Lauwereins, Thoen et al.
        \newblock \emph{DCASE Challenge 2018 Task 5}, 2018.
        \newblock \url{http://dcase.community/challenge2018/task-monitoring-domestic-activities}
    \end{thebibliography}
\end{frame}

% This is an example frame for how to work with animations in latex
\begin{frame}[fragile]{Example slide with animations}
    \begin{lstlisting}[
        linebackgroundcolor={%
        \btLstHL<1>{1-3}%
        \btLstHL<2>{6,9}%
        \btLstHL<3>{7}%
        \btLstHL<4>{8}%
        },
        gobble=4,
        label={lst:test}]
      /**
      * Prints Hello World.
      **/
      #include <stdio.h>

      int main(void) {
         printf("Hello World!");
         return 0;
      }
    \end{lstlisting}

    \only<1>{first}
    \only<1-2>{first-second}
    \only<3->{from-3}
    \only<2>{second}
    \onslide<3>{third}
    \onslide<4>{fourth}
\end{frame}
