% Borrowing & Move-Semantik

% Example direct code, note the [fragile] annotation
\begin{frame}[fragile,t]{Zuweisung einer Ganzzahl}
    \begin{lstlisting}[language=Rust,escapechar=@,label={lst:borrowing1}]
fn main() {
    let var1 = 1;
    let var2 = var1;

    println!("Die Werte sind: {} und {}", var1, var2);
}\end{lstlisting}
    \pause
    \codeoutput{code/02-borrowing1.txt}
    \note{
        Einfache werte können direkt einer neuen variable zugewiesen werden.
        Wie dies auch in Java möglich ist.
        Beim Verlassen der \texttt{main()} wird \texttt{var1} und \texttt{var2} aufgeräumt.
    }
\end{frame}

% Example direct code, note the [fragile] annotation
\begin{frame}[fragile,t]{Zuweisung eines Strings}
    \begin{lstlisting}[
        language=Rust,escapechar=@,label={lst:borrowing2},
        linebackgroundcolor={%
        \btLstHL<1>{}% No highlighting on first slide
        \btLstHL<2>{5}% Highlight second slide
        },
    ]
fn main() {
    let var1 = String::from("hello");
    let var2 = var1;

    println!("Die Werte sind: {} und {}", var1, var2);
}\end{lstlisting}
    \pause
    \codeoutput[red]{code/02-borrowing2.txt}
    \note{
        Obwohl ähnlich zum beispiel zuvor, sieht Rust \texttt{var1} nach zeile 3 nicht mehr als gültig an. \\
        Wie bei Java ist ein String in Rust ein komplexer Datentyp. \\
        Wie zuvor werden die Werte, wenn die main verlassen wird aufgeräumt. \\
        Wenn die Zuordnung so funktionieren würde, müsste der String zweimal bereinigt werden.
    }
\end{frame}

% Example direct code, note the [fragile] annotation
\begin{frame}[fragile,t]{String an funktion übergeben I}
    \begin{lstlisting}[
        language=Rust,escapechar=@,label={lst:borrowing3},
        linebackgroundcolor={%
        \btLstHL<1>{}% No highlighting on first slide
        \btLstHL<2>{7}% Highlight second slide
        },
    ]
fn main() {
    let s1 = String::from("hello");
    let len = calculate_length(s1);
    println!("Die Länge von '{}' ist {}.", s1, len);
}

fn calculate_length(s: String) -> usize {
    s.len()
}\end{lstlisting}
    \pause
    \codeoutput[red]{code/02-borrowing3.txt}
\end{frame}

\begin{frame}[fragile,t]{String an funktion übergeben II}
    \begin{lstlisting}[
        language=Rust,escapechar=@,label={lst:borrowing4},
        linebackgroundcolor={%
        \btLstHL<1>{3,7}% No highlighting on first slide
        \btLstHL<2>{}% Highlight second slide
        },
    ]
fn main() {
    let s1 = String::from("hello");
    let len = calculate_length(&s1);
    println!("Die Länge von '{}' ist {}.", s1, len);
}

fn calculate_length(s: &String) -> usize {
    s.len()
}\end{lstlisting}
    \only<1>{
        Übergabe einer Referenz (\texttt{\&})
        \note{
            Übergabe einer Referenz (\texttt{\&}) löst das Problem. \\
            Allerdings wird hier garantiert, dass die Referenz bis zum Ende der Laufzeit valide ist. \\
            Anders als dies bei Pointern in C der Fall ist.
        }
    }
    \pause
    \codeoutput{code/02-borrowing4.txt}
\end{frame}
