\documentclass[letterpaper,12pt]{article}
\usepackage{tabularx} % extra features for tabular environment
\usepackage{amsmath}  % improve math presentation
\usepackage{graphicx} % takes care of graphic including machinery
\usepackage[margin=1in,letterpaper]{geometry} % decreases margins
\usepackage{cite} % takes care of citations
\usepackage[final]{hyperref} % adds hyper links inside the generated pdf file
\hypersetup{
    colorlinks=true,       % false: boxed links; true: colored links
    linkcolor=blue,        % color of internal links
    citecolor=blue,        % color of links to bibliography
    filecolor=magenta,     % color of file links
    urlcolor=blue
}
\usepackage{blindtext}
%++++++++++++++++++++++++++++++++++++++++

% Do not indent second paragraph
\setlength{\parindent}{0pt}

\begin{document}

    \title{Programming Concepts \& Paradigms\\Rust}
    \author{Roman Schilter \& Jan-Henrik Preuß\\[0.4cm]{\small Betreuer: Marcel Baumann \& Ruedi Arnold}}
    \date{May 31, 2024}
    \maketitle

%Wenige Seiten (ca. 2-4) reichen durchaus, max. 5 (bei mehr
%gibt's tendenziell Abzug), Inhaltsverzeichnis nicht nötig

    \begin{abstract}
        % Beschreibe das dokument knapp
        Rust ist eine moderne Programmiersprache, die Sicherheit und Geschwindigkeit vereint.
        Sie wurde von Mozilla 2010 entwickelt und ist Open Source.
        Rust hat den Anspruch, die Lücke zwischen Low-Level-Programmiersprachen wie C und C++ und High-Level-Programmiersprachen wie Java und Python zu schließen.
        Dieses Dokument fasst die wichtigsten Konzepte und Paradigmen von Rust im Vergleich zu Java zusammen.
    \end{abstract}

%    Fokus auf wichtige/ interessante/ spezielle Sprach-
%Eigenschaften! (Was anders als bei Java?!...)
%§ Ergänzend zu den Folien
%– Falls interessant/relevant kurze Infos zu Vision,
%Geschichte & Verbreitung
%– Hauptteil: Die Sprache vorstellen (Ihre 3 bis 7
%Fokuspunkte, inkl. Verweise auf Ihren Demo-Code)


    \section{Sprachkonzepte}

    Hier werden die wichtigsten Sprachkonzepte von Rust.

    \blindtext %delete this line

    \subsection{Borrowing \& Move-Semantik}\label{subsec:borrowing-&-move-semantik}

    \subsection{Traits: bounds \& associated types}\label{subsec:traits:-bounds-&-associated-types}

    \subsection{Typestate Programming}\label{subsec:typestate-programming)}

    \subsection{Tasks, Communication, Spawn \& Channels}\label{subsec:tasks-communication-spawn-&-channels}

    \subsection{Patterns \& Matching}\label{subsec:patterns-&-matching}

    \subsection{Cargo: Test \& Build}\label{subsec:cargo:-test-&-build}

%    Ihr technisches Team-Fazit


    \section{Team-Fazit}\label{sec:team-fazit}

    \blindtext %delete this line

% Persönliches Fazit (je min. 1 Abschnitt pro Team-Mitglied)


    \section{Persönliches Fazit}\label{sec:personliches-fazit}

    \subsection{Roman Schilter}\label{subsec:roman}

    \subsection{Jan-Henrik Preuß}\label{subsec:jan}

%++++++++++++++++++++++++++++++++++++++++
% References section will be created automatically
% with inclusion of "thebibliography" environment
% as it shown below. See text starting with line
% \begin{thebibliography}{99}
% Note: with this approach it is YOUR responsibility to put them in order
% of appearance.

%    \begin{thebibliography}{99}
%
%        \bibitem{melissinos}
%        A.~C. Melissinos and J. Napolitano, \textit{Experiments in Modern Physics},
%        (Academic Press, New York, 2003).
%
%        \bibitem{Cyr}
%        N.\ Cyr, M.\ T$\hat{e}$tu, and M.\ Breton,
%% "All-optical microwave frequency standard: a proposal,"
%        IEEE Trans.\ Instrum.\ Meas.\ \textbf{42}, 640 (1993).
%
%        \bibitem{Wiki} \emph{Expected value},  available at
%        \texttt{http://en.wikipedia.org/wiki/Expected\_value}.
%
%    \end{thebibliography}


\end{document}
