\documentclass[letterpaper,12pt]{article}
\usepackage{tabularx} % extra features for tabular environment
\usepackage{amsmath}  % improve math presentation
\usepackage{graphicx} % takes care of graphic including machinery
\usepackage[margin=1in,letterpaper]{geometry} % decreases margins
\usepackage{cite} % takes care of citations
\usepackage[final]{hyperref} % adds hyper links inside the generated pdf file
\hypersetup{
    colorlinks=true,       % false: boxed links; true: colored links
    linkcolor=blue,        % color of internal links
    citecolor=blue,        % color of links to bibliography
    filecolor=magenta,     % color of file links
    urlcolor=blue
}
\usepackage{blindtext}
%++++++++++++++++++++++++++++++++++++++++

% Do not indent second paragraph
\setlength{\parindent}{0pt}

\begin{document}

    \title{Programming Concepts \& Paradigms\\Rust}
    \author{Roman Schilter \& Jan-Henrik Preuß\\[0.4cm]{\small Betreuer: Marcel Baumann \& Ruedi Arnold}}
    \date{May 31, 2024}
    \maketitle

    \begin{abstract}
    \end{abstract}

    \section{Introduction}

    Motivate why you chose the problem that you did. Why is it interesting?

    \blindtext %delete this line


    \section{Background}

    \blindtext %delete this line

    \section{Methods}

    \section{Results}


    \blindtext


    \section{Conclusions}

    \blindtext

    \section{Future Work}

    \blindtext

%++++++++++++++++++++++++++++++++++++++++
% References section will be created automatically
% with inclusion of "thebibliography" environment
% as it shown below. See text starting with line
% \begin{thebibliography}{99}
% Note: with this approach it is YOUR responsibility to put them in order
% of appearance.

%    \begin{thebibliography}{99}
%
%        \bibitem{melissinos}
%        A.~C. Melissinos and J. Napolitano, \textit{Experiments in Modern Physics},
%        (Academic Press, New York, 2003).
%
%        \bibitem{Cyr}
%        N.\ Cyr, M.\ T$\hat{e}$tu, and M.\ Breton,
%% "All-optical microwave frequency standard: a proposal,"
%        IEEE Trans.\ Instrum.\ Meas.\ \textbf{42}, 640 (1993).
%
%        \bibitem{Wiki} \emph{Expected value},  available at
%        \texttt{http://en.wikipedia.org/wiki/Expected\_value}.
%
%    \end{thebibliography}


\end{document}
